\documentclass[12pt,a4paper]{article}
\usepackage{geometry}
\usepackage{slashbox}
\geometry{
	a4paper,
	total={170mm,257mm},
	left=20mm,
	right=20mm,
	top=20mm,
	bottom=30mm
}
\usepackage{graphicx}
\usepackage{pdfpages}
\usepackage{placeins}
\usepackage{float}

\usepackage{polski}
\usepackage[utf8]{inputenc}

\usepackage{listings}
\usepackage{lmodern}
\lstdefinelanguage
[x64]{Assembler}     % add a "x64" dialect of Assembler
[x86masm]{Assembler} % based on the "x86masm" dialect
% with these extra keywords:
{morekeywords={CDQE,CQO,CMPSQ,CMPXCHG16B,JRCXZ,LODSQ,MOVSXD, %
		POPFQ,PUSHFQ,SCASQ,STOSQ,IRETQ,RDTSCP,SWAPGS, %
		rax,rdx,rcx,rbx,rsi,rdi,rsp,rbp, %
		r8,r8d,r8w,r8b,r9,r9d,r9w,r9b, %
		r10,r10d,r10w,r10b,r11,r11d,r11w,r11b, %
		r12,r12d,r12w,r12b,r13,r13d,r13w,r13b, %
		r14,r14d,r14w,r14b,r15,r15d,r15w,r15b}} % etc.
\lstset
{ %Formatting for code in appendix
	basicstyle=\footnotesize,
	language=[x64]Assembler,
	numbers=left,
	stepnumber=1,
	tabsize=4,
	basicstyle=\ttfamily,
}

\begin{document}
	
	\begin{titlepage}
		\newgeometry{top=5.5cm, bottom=3cm}
		
		\centering
		{\huge\bfseries Podstawy techniki mikroprocesorowej lab.\par}
		
		\vspace{0.5cm}
		Prowadzący: Mgr inż. Antoni Sterna (E00-74ap, środa 13:15) \\
	
		\vspace{1.1cm}
		{\Large Sprawozdanie 2 - 2018.03.21\par}
		\vfill
		
		{\large\bfseries Jakub Dorda 235013\par}
		{\large\bfseries Marcin Kotas 235098\par}
		
		\vspace{1cm}
		\today \\ \LaTeX
		
		\restoregeometry
	\end{titlepage}

	\newgeometry{top=1cm, bottom=1.2cm, left=20mm, right=20mm}

	\section{Wprowadzenie/cel ćwiczeń}
		Celem ćwiczeń było wykonanie dwóch programów mających na celu przećwiczenie obsługi zegara oraz przerwań prze niego wywoływanych. Pierwszy program polegał na przepisaniu kodu z poprzednich zajęć zastępując opóźnienie programowe - wielokrotne wywołanie nop przez czas zliczany na rejestrze zegara. Drugi program miał na celu przećwiczenie obsługi przerwań oraz zmiennych.
		
	\section{Kod programu pierwszego}
		\begin{minipage}{.5\textwidth}
			\lstinputlisting{../../lab2/zad1.asm}
		\end{minipage}%
		\begin{minipage}{.5\textwidth}
			Stała LOAD oraz pomocnicze TIME i CYKLE, służą do wyliczenia zakresu ustawianego potem na rejestrach tl0, th0,
			niezbędne do poprawnej konfiguracji zegara.\\\\
			Podstawowa część programu nie została zmieniona, polega na przesuwaniu bitów w akumulatorze w efekcie otrzymujemy efekt
			poruszającej się diody w prawo oraz lewo z odbiciem.\\\\
			Subrutyna "timer\_init" konfiguruje tryb działania zegara. Flaga tr0 przedstawia stan zegara. Ustawiony bit sygnalizuje działanie zegara. \\\\
			Subrutyna "delay" wywołuje się sama do momentu przekroczenia zakresu rejestru liczącego impulsy zegara. Flaga tf0 to flaga przepełnienia. Jest automatycznie ustawiana w momencie, gdy zegar przekroczy zakres 16 bitów (dla trybu 16-bitowego).
		\end{minipage}
	
	\section{Kod programu drugiego}
		\begin{minipage}{.5\textwidth}
			\lstinputlisting[firstnumber=1, firstline=1, lastline=45]{../../lab2/zad2.asm}
		\end{minipage}%
		\begin{minipage}{.5\textwidth}
			Drugi program wykorzystuje przerwania określone w 0BH do zwiększania zmiennej LICZNIK, po zmniejszeniu wartości z 20 do 0 odmierza czas 1 sekundy. Po odliczeniu 1s przez LICZNIK zwiększamy SECONDS
			sprawdzając jednocześnie czy nie przekroczono wartości 60, jeśli tak się stało skaczemy do etykiety "minute" i zwiększamy zmienną MINUTES o 1 jednocześnie resetując SECONDS na 0. W przypadku przekroczenia przez MINUTES wartości 60, skaczemy do "hour" i zwiększamy HOURS o 1 jednocześnie resetując MINUTES do 0.
			Sprawdzamy czy zmienna nie przekroczyła wartości 24, jeśli tak to ustawiamy jej wartość na 0.\\\\
		\end{minipage}
		
		\begin{minipage}{.5\textwidth}
			\lstinputlisting[firstnumber=46, firstline=46, lastline=100]{../../lab2/zad2.asm}
		\end{minipage}%
		\begin{minipage}{.5\textwidth}
			Subrutyna "timer" zawiera konfigurację zegara, oraz obsługi przerwania. ustawienie bitu na rejestrze ea włącza obsługę
			przerwań, et0 włącza wywoływanie przerwania przy przepełnienie rejestru zegara.
		\end{minipage}

	\section{Wnioski/podsumowanie}
	
			Wszystkie programy uruchomiły się poprawnie. Nie udało się w czasie laboratorium skończyć programu mającego na celu konwersję czasu pobieranego z zegara na format HH:MM:SS, został on dokończony poza czasem trwania zajęć.
	
\end{document}