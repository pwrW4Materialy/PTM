\documentclass[12pt,a4paper]{article}
\usepackage{geometry}
\usepackage{slashbox}
\geometry{
	a4paper,
	total={170mm,257mm},
	left=20mm,
	right=20mm,
	top=20mm,
	bottom=20mm
}
\usepackage{graphicx}
\usepackage{pdfpages}
\usepackage{placeins}
\usepackage{float}

\usepackage{polski}
\usepackage[utf8]{inputenc}
\usepackage{lmodern}

\usepackage{listings}
\lstset{
	language=[x86masm]Assembler,
	breaklines=true,
	tabsize=4,
	basicstyle=\ttfamily,
}


\begin{document}
	
	\begin{titlepage}
		\newgeometry{top=5.5cm, bottom=3cm}
		
		\centering
		{\huge\bfseries Podstawy techniki mikroprocesorowej lab.\par}
		
		\vspace{0.5cm}
		Prowadzący: Mgr inż. Antoni Sterna (E00-74ap, środa 13:15) \\
	
		\vspace{1.1cm}
		{\Large sprawozdanie 3 - 2018.04.04\par}
		\vfill
		
		{\large\bfseries Jakub Dorda 235013\par}
		{\large\bfseries Marcin Kotas 235098\par}
		
		\vspace{1cm}
		\today \\ \LaTeX
		
		\restoregeometry
	\end{titlepage}

	\newgeometry{top=1.5cm, bottom=1.5cm, left=20mm, right=20mm}

	\section{Wprowadzenie/cel ćwiczeń}
		Celem ćwiczeń było napisanie funkcji obsługujących wyświetlacz LCD\@.
		
	\section{Realizacja funkcji}
		\subsection{Zadeklarowane stałe}
			\begin{minipage}{.5\textwidth}
				\lstinputlisting[firstline=0,lastline=7]{../../lab3/lab3.asm}
			\end{minipage}%
			\begin{minipage}{.5\textwidth}
				Zmienne deklarują ustawienia odpowiednich adresów w pamięci, które mapowane są do obsługi wyświetlacza.
				Zmienna \texttt{MODE} deklaruje tryb działania wyświetlacza (dwie linijki oraz przesuwanie wskaźnika na tekst wraz z pisaniem)
			\end{minipage}
		
		\subsection{Inicjalizacja wyświetlacza}
			\begin{minipage}{.5\textwidth}
				\lstinputlisting[firstline=26,lastline=33]{../../lab3/lab3.asm}
			\end{minipage}%
			\begin{minipage}{.5\textwidth}
				Funkcja \texttt{init\_lcd} uruchamia i czyści wyświetlacz, a następnie ustawia tryb działania.
			\end{minipage}

		\subsection{Wysyłanie rozkazów do wyświetlacza}
			\begin{minipage}{.5\textwidth}
				\lstinputlisting[firstline=35,lastline=41]{../../lab3/lab3.asm}
			\end{minipage}%
			\begin{minipage}{.5\textwidth}
				Funkcja \texttt{init\_cmd} wysyła kod rozkazu z akumulatora na odpowiedni adres.
				Przed każdym wysłaniem rozkazu następuje sprawdzenie, czy wyświetlacz nie wykonuje w tym momencie innego rozkazu.
			\end{minipage}

		\subsection{Sprawdzanie stanu wyświetlacza}
			\begin{minipage}{.5\textwidth}
				\lstinputlisting[firstline=90,lastline=95]{../../lab3/lab3.asm}
			\end{minipage}%
			\begin{minipage}{.5\textwidth}
				Funkcja \texttt{wait\_busy} czyta status wyświetlacza z adresu \texttt{RD\_STATUS}, a następnie sprawdza, czy ustawiona jest bit siódmy (busy flag).
				Sprawdzanie następuje w pętli aż flaga nie będzie ustawiona. 
			\end{minipage}

		\subsection{Wypisywanie znaku na wyświetlacz}
			\begin{minipage}{.5\textwidth}
				\lstinputlisting[firstline=43,lastline=49]{../../lab3/lab3.asm}
			\end{minipage}%
			\begin{minipage}{.5\textwidth}
				Funkcja \texttt{lcd\_writec} pobiera kod znaku z akumulatora, upewnia się, że wyświetlacz nie jest zajęty, a następnie wysyła kod znaku pod adres \texttt{WR\_DATA}.
			\end{minipage}

		\subsection{Wypisywanie tekstu na wyświetlacz}
			\begin{minipage}{.5\textwidth}
				\lstinputlisting[firstline=51,lastline=65]{../../lab3/lab3.asm}
			\end{minipage}%
			\begin{minipage}{.5\textwidth}
				Funkcja \texttt{lcd\_writestr} pobiera adres do tekstu z rejestru DPTR, a następnie kopiuje jeden znak do akumulatora.
				Jeżeli pobrany do akumulatora znak jest równy 0, to cały tekst został już wypisany i funkcja kończy działanie.

				W pozostałych przypadkach funkcja odkłada na stos obecny wskaźnik w DPTR i wywołuje funkcję \texttt{lcd\_writec}.
				Następnie przywraca poprzedni wskaźnik na tekst do DPTR, zwiększa go o 1 i wraca na początek funkcji.
			\end{minipage}

		\subsection{Manipulowanie pozycją wskaźnika tekstu}
			\begin{minipage}{.5\textwidth}
				\lstinputlisting[firstline=67,lastline=76]{../../lab3/lab3.asm}
			\end{minipage}%
			\begin{minipage}{.5\textwidth}
				Funkcja \texttt{lcd\_gotoxy} pobiera wybrany adres wskaźnika z akumulatora w następującym formacie: najmłodsze 4 bity oznaczają pozycję x, bit na pozycji czwartej oznacza rząd (0 lub 1).
				Dzięki temu bity w pierwszym rzędzie adresowane są od 0 do 15, a bity w drugim rzędzie od 16 do 31.

				Funkcja najpierw zeruje wszystkie bity 5-7 tak, aby ograniczyć wartość do 31 w przypadku błędnego argumentu.
				Następnie sprawdza, czy bit czwarty jest ustawiony i jeśli tak, to najpierw go zeruje, a później dodaje do całej wartości \texttt{40H}.
				Dzieje się tak ponieważ w rejestrze danych wyświetlacza rząd drugi jest umiejscowiony w adresach od \texttt{40H}.

				Po uzyskaniu końcowego adresu funkcja ustawia bit siódmy, co jest informacją dla wyświetlacza, że młodsze siedem bitów jest adresem wskaźnika pamięci danych.
				W tej postaci rozkaz przekazywany jest do funkcji \texttt{lcd\_cmd}.
			\end{minipage}
			
		\subsection{Zamiana liczb na tekst}
			\begin{minipage}{.5\textwidth}
				\lstinputlisting[firstline=78,lastline=88]{../../lab3/lab3.asm}
			\end{minipage}%
			\begin{minipage}{.5\textwidth}
				Funkcja \texttt{int\_to\_str} pobiera liczbę z akumulatora, a następnie dzieli ją przez 10.
				Po tej operacji w akumulatorze jest wynik dzielenia, a w rejestrze B jest reszta z dzielenia.
				Do każdej z tych wartości dodawany jest kod ASCII znaku 0, a następnie funkcja zwraca przez stos oba znaki.
				Funkcja najpierw odkłada na stos resztę z dzielenia, a następnie wynik, aby po wywołaniu funkcji znaki mogły być pobierane ze stosu i przekazywane do funkcji \texttt{lcd\_writec} w odpowiedniej kolejności.
			\end{minipage}

	\section{Wnioski/podsumowanie}
		Wszystkie funkcje zrealizowane podczas zajęć uruchomiły się poprawnie.
		Funkcja zamiany liczb na tekst została dopisana po zajęciach.
	
\end{document}