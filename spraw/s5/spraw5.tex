\documentclass[12pt,a4paper]{article}
\usepackage{geometry}
\usepackage{slashbox}
\geometry{
	a4paper,
	total={170mm,257mm},
	left=20mm,
	right=20mm,
	top=20mm,
	bottom=30mm
}
\usepackage{graphicx}
\usepackage{pdfpages}
\usepackage{placeins}
\usepackage{float}

\usepackage{polski}
\usepackage[utf8]{inputenc}

\usepackage{listings}
\usepackage{lmodern}
\lstdefinelanguage
[x64]{Assembler}     % add a "x64" dialect of Assembler
[x86masm]{Assembler} % based on the "x86masm" dialect
% with these extra keywords:
{morekeywords={CDQE,CQO,CMPSQ,CMPXCHG16B,JRCXZ,LODSQ,MOVSXD, %
		POPFQ,PUSHFQ,SCASQ,STOSQ,IRETQ,RDTSCP,SWAPGS, %
		rax,rdx,rcx,rbx,rsi,rdi,rsp,rbp, %
		r8,r8d,r8w,r8b,r9,r9d,r9w,r9b, %
		r10,r10d,r10w,r10b,r11,r11d,r11w,r11b, %
		r12,r12d,r12w,r12b,r13,r13d,r13w,r13b, %
		r14,r14d,r14w,r14b,r15,r15d,r15w,r15b}} % etc.
\lstset
{ %Formatting for code in appendix
	basicstyle=\footnotesize,
	language=[x64]Assembler,
	numbers=left,
	stepnumber=1,
	tabsize=4,
	basicstyle=\ttfamily,
}

\begin{document}
	
	\begin{titlepage}
		\newgeometry{top=2.5cm, bottom=2.5cm}
		
		\centering
		{\huge\bfseries Podstawy techniki mikroprocesorowej lab.\par}
		
		\vspace{0.5cm}
		Prowadzący: Mgr inż. Antoni Sterna (E00-74ap, środa 13:15) \\
	
		\vspace{1.1cm}
		{\Large Sprawozdanie 5 - 2018.05.21\par}
		\vfill
		
		{\large\bfseries Jakub Dorda 235013\par}
		{\large\bfseries Marcin Kotas 235098\par}
		
		\vspace{1cm}
		\today \\ \LaTeX
		
		\restoregeometry
	\end{titlepage}

	\newgeometry{top=1cm, bottom=1.2cm, left=20mm, right=20mm}

	\section{Wprowadzenie/cel ćwiczeń}
		Celem ćwiczeń była nauka obsługi klawiatury alfanumerycznej.
		
	\section{Kod programu}
		Rejestry \verb|P5| oraz \verb|P7| w używanym zestawie podłączone są odpowiednio pod adresami \verb|0F8H| i \verb|0DBH|.
		
		\begin{minipage}{.5\textwidth}
			\lstinputlisting[firstnumber=1, firstline=1, lastline=9]{../../lab5/lab5.asm}
		\end{minipage}%	
		\begin{minipage}{.5\textwidth}
			W głównej pętli programu następuje wywoływanie funkcji \verb|read_keyboard|,
			która, jeśli któryś z przycisków był naciśnięty, ustawia bit \verb|C| oraz zwraca kod klawisza w akumulatorze.
			
			Jeżeli bit \verb|C| jest ustawiony, następuje skok do etykiety \verb|is_key|.
			W przeciwnym przypadku do rejestru \verb|P1| wstawiana jest wartość \verb|0ffh|, która wygasza wszystkie diody.
		\end{minipage}
		
		\begin{minipage}{.5\textwidth}
			\lstinputlisting[firstnumber=12, firstline=12, lastline=36]{../../lab5/lab5.asm}
		\end{minipage}%	
		\begin{minipage}{.5\textwidth}
			Procedura \verb|read_keyboard| sprawdza po kolei kolejne wiersze poprzez wywoływanie funkcji \verb|read_row|.
			W rejestrze \verb|R1| znajduje się numer wiersza (bit 7 = pierwszy wiersz, bit 4 = ostatni).
			Młodsze bity \verb|R1| ustawione są na \verb|1|, aby umożliwić zmianę numeru wiersza za pomocą komendy \verb|RR|.

			Pętla \verb|read_kb| wykonuje się 4 razy, aby sprawdzić każdy wiersz.
			Funkcja \verb|read_row| ustawia bit \verb|C| oraz zwraca numer kolumny w akumulatorze, jeżeli któryś z przycisków był naciśnięty.
			
			Jeżeli wywołanie \verb|read_row| ustawiło bit \verb|C|, następuje skok do etykiety \verb|found_key|.
			Jeżeli w żadnej iteracji pętli nie został ustawiony bit \verb|C|, funkcja wraca do głównej pętli programu.

			Na początku procedury \verb|found_key| następuje odłożenie rejestru \verb|PSW| na stos, aby zachować stan flagi \verb|CY|.
			Następnie wartość rejestru \verb|R0| (numer wiersza) wymnażana jest przez 4, a następnie dodawany jest numer kolumny.
			W ten sposób uzyskiwany jest unikalny kod wciśniętego przycisku.
		\end{minipage}
		
		\begin{minipage}{.5\textwidth}
			\lstinputlisting[firstnumber=38, firstline=38, lastline=63]{../../lab5/lab5.asm}
		\end{minipage}%	
		\begin{minipage}{.5\textwidth}
			Funkcja \verb|read_row| na początku zeruje górną połowę rejestru \verb|P5| (na tych bitach podłączone są wiersze klawiatury).
			Następnie zeruje dolną połowę akumulatora (aby nie nadpisać bitów w dolnej połowie \verb|P5|)
			oraz ustawia wybrane bity poprzez operację \verb|ORL P5, A|.

			W następnej kolejności kopiuje zawartość rejestru \verb|P7|
			do akumulatora oraz zeruje starszą połowę (4 młodsze bity \verb|P7| informują,
			w której kolumnie przycisk został wciśnięty).

			W pętli \verb|columns| następuje przesuwanie z przeniesieniem w prawo \verb|A| do momentu ustawienia bitu \verb|C|.
			Licznik pętli \verb|R2| informuje o numerze kolumny, w której został naciśnięty przycisk.
		\end{minipage}

		\begin{minipage}{.5\textwidth}
			\lstinputlisting[firstnumber=66, firstline=66, lastline=70]{../../lab5/lab5.asm}
		\end{minipage}%	
		\begin{minipage}{.5\textwidth}
			W procedurze \verb|is_key| następuje wyzerowanie bitu \verb|ACC.7|,
			aby zasygnalizować, że na diodach wyświetlany jest kod przycisku (w przypadku przycisku o kodzie 0 nie zapalą się żadne inne diody)
			Następnie zawartość akumulatora kopiowana jest do rejestru \verb|P1|,
			do którego podłączone są diody, po czym następuje powrót do pętli głównej programu.
		\end{minipage}
		
	\section{Wnioski/podsumowanie}
	
			Program uruchomił się poprawnie.
	
\end{document}