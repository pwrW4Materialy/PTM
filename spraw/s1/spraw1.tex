\documentclass[12pt,a4paper]{article}
\usepackage{geometry}
\usepackage{slashbox}
\geometry{
	a4paper,
	total={170mm,257mm},
	left=20mm,
	right=20mm,
	top=20mm,
	bottom=20mm
}
\usepackage{graphicx}
\usepackage{pdfpages}
\usepackage{placeins}
\usepackage{float}

\usepackage{polski}
\usepackage[utf8]{inputenc}

\usepackage{listings}
\lstset{
	language=[x86masm]Assembler,
	breaklines=true,
	tabsize=4,
}


\begin{document}
	
	\begin{titlepage}
		\newgeometry{top=5.5cm, bottom=3cm}
		
		\centering
		{\huge\bfseries Podstawy techniki mikroprocesorowej lab.\par}
		
		\vspace{0.5cm}
		Prowadzący: Mgr inż. Antoni Sterna (E00-74ap, środa 13:15) \\
	
		\vspace{1.1cm}
		{\Large sprawozdanie 1 - 2018.03.21\par}
		\vfill
		
		{\large\bfseries Jakub Dorda 235013\par}
		{\large\bfseries Marcin Kotas 235098\par}
		
		\vspace{1cm}
		\today \\ \LaTeX
		
		\restoregeometry
	\end{titlepage}

	\newgeometry{top=1.5cm, bottom=1.5cm, left=20mm, right=20mm}

	\section{Wprowadzenie/cel ćwiczeń}
		Celem ćwiczeń było wykonanie trzech programów. Pierwszy program kopiował dane z pamięci wewnętrznej do zewnętrznej, drugi sterował diodami, a trzeci kopiował dane z pamięci zewnętrznej do zewnętrznej.
		
	\section{Kod programu pierwszego}
		\begin{minipage}{.5\textwidth}
			\lstinputlisting{../../lab1/prg1.s}
		\end{minipage}%
		\begin{minipage}{.5\textwidth}
			Aby skopiować do zewnętrznej pamięci należy umieścić adres komórki w 16 bitowym rejestrze DPTR.
			Przy korzystaniu z tego rejestru należy użyć komendy movx zamiast mov.
		\end{minipage}
	
	\section{Kod programu drugiego}
	\begin{minipage}{.5\textwidth}
		\lstinputlisting{../../lab1/prg2.a51}
	\end{minipage}%
	\begin{minipage}{.5\textwidth}
		Drugi program wykorzystuje funkcję nop (no operation) aby wymusić opóźnienie wykonywania programu. Wykonanie pętli d0 trwa w przybliżeniu 4 us. Poprzez dodanie dodatkowych liczników w r0 i r1, pętla delay wykonuje się 10ms*r3.
		W naszym przypadku w r3 znajduje się 50, więc pętla wykonuje się pół sekundy.
		
		Sterowanie diodami odbywa się za pomocą rejestru p1. Dla 0 dioda się świeci, dla 1 nie. Efekt przesuwającej się diody uzyskany został przez użycie funkcji rr oraz rl, które przesuwają odpowiednio w prawo oraz w lewo bity znajdujące się w akumulatorze.
	\end{minipage}

	\section{Kod programu trzeciego}
	\begin{minipage}{.5\textwidth}
		\lstinputlisting{../../lab1/prg3.a51}
	\end{minipage}%
	\begin{minipage}{.5\textwidth}
		Aby program mógł kopiować dane z pamięci zewnętrznej do zewnętrznej, należy użyć dwa razy rejestru DPTR. W tym celu na początku pętli do DPTR wstawiany jest adres z rejestrów r0 i r1, z którego mają zostać skopiowane dane. Wtedy odbywa się kopiowanie do akumulatora zawartości pamięci pod adresem wskazywanym przez DPTR oraz zwiększenie adresu DPTR. Następnie do rejestrów r0 i r1 odstawiany jest aktualny adres DPTR oraz wstawiany nowy adres zachowany w rejestrach r2 i r3. Dopiero wtedy kopiuje zawartość akumulatora pod nowy adres wskazywany przez DPTR, oraz zwiększa adres DPTR. Na końcu pętli adres wskazywany prze dptr jest wstawiany z powrotem do rejestrów r2 i r3.
	\end{minipage}

	\section{Wnioski/podsumowanie}
	
			Wszystkie programy uruchomiły się poprawnie.
	
\end{document}